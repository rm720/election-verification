\chapter{Results}
Upon successful completion, the project provides the following significant contributions:

\section{Technique Improvement} 
We enhanced the technique  proposed by Haines et. al. \cite{Haines2019VerifiedVF}, by offering an equivalent 
counterpart that resides in HOL environment instead of Coq proof assistant. The benefits of such relocation
 is that resulting executable of election verifier achieves end-to-end correctness. The reason is the following. 
 The technique includes logical building blocks and proofs that have to be used to develop election verifier, 
 since these all now in HOL, the resulting election verifier implementation will also be in HOL. Code from HOL 
 can be compiled with CakeML compiler, producing guaranteed correct executable program. Such technology is not 
 possible in Coq proof assistant, the previous residence of the technique. Thus, our enhancement, pushed election 
 verifier correctness forward and consequently promoted electronic election trustworthiness.

\section{Proof-Based Development}
 We contribute to the maturity of proof-based software development by practicing 
 it while working in HOL and CakeML on the technique improvement. These tools, 
 besides of offering exceptional correctness opportunities, are still work in 
 progress. Using such underdeveloped tools is tidies but rewarding. During our 
 work we revealed limitations of the above tools, which give an opportunity for 
 improvement. Specifically HOL4 packages system has duplicating Ring Theories. One is 
 named IntegerRing\footnote{https://github.com/HOL-Theorem-Prover/HOL/blob/develop/src/integer/integerRingScript.sml} and 
 another numRing\footnote{https://github.com/HOL-Theorem-Prover/HOL/blob/develop/src/ring/src/numRingScript.sml}. 
 These competing Ring theories prevent from importing theory that uses Ring theory. 
 The issue can be temporarily patched by disabling one of packages in the local built, however, 
 currently this is a limitation. In addition, CakeML programming language is not able to open and read 
 JSON file, because required library is has not yet been developed.

