\chapter{Analysis}

\section{Technique Improvement} 
The main objective of this project was to improve the previously presented technique for developing election verifiers (Haines et al., 2019). In order to evaluate an improvement, we need to evaluate an improved technique. A technique can be evaluated by applying it to develop an election verifier and running it to verify an election.
In our case, we successfully developed an election verifier. However, we were unable to verify a real election due to limitations with the CakeML libraries, which were unable to read JSON files. As a result, the evaluation of our technique was not fully successful.
However, we do not consider the project a failure. Our technique is still usable and capable of developing a proven correct election verifier program that is ready to be compiled.

The impact of our improvement is that it brought election verifiers closer to end-to-end correctness. This leads to greater confidence in the results of election verification, and in the long term, it will entail better trustworthiness of electronic elections.

\section{Election Verifier}
Since our work includes a demonstration of the technique the resukt of this demonstration is proven correct election verifier. This result is represented by pieces of code. The code is traditionally evaluated by running and testing, however, evaluating the quality of proven correct programs can be challenging. The reason for this is that these programs are proven correct before tests are run, meaning that they are correct by construction. Consequently, traditional tools for measuring software quality, such as testing and evaluations, are weaker than proof and fall short in providing evaluations. As a result, the best evaluation of our code is conducted through peer review by ANU Scholars, HOL, CakeML, Helios communities, and other interested parties.

The impact of our work is realisation that it is feasible to develop an end-to-end correct election verifier using verified tools such as HOL4 and CakeML. However, the cost of doing so is higher than it might seem. This is because proof-based development is not yet a mature technology. There are real challenges in unexpected places.


\section{Proof-Based Development}
In this sceptre our work appear like just some program that needs to be proven for correct operation. Since we were working on it, our work is inevitably contributes to maturing and popularising of proof-based software development. 
To evaluate the results in this direction, we can look at the obstacles and limitations we discovered on the way to achieving complete correctness of operational programs. The evaluation of this result is rather modest, since we only discovered a couple of limitations and advertised HOL and CakeML to the people we know. 

The impact of these discoveries is that tools such as HOL4 and CakeML will become better and attract more users. These users, in turn, will lead to further improvements, leading towards a more accessible proof-based software development for people.



















